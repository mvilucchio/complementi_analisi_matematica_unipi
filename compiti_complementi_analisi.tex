\documentclass[a4paper]{article}
\usepackage[T1]{fontenc}
\usepackage[utf8]{inputenc}
\usepackage[italian]{babel}
\usepackage{microtype}
\usepackage{amsmath}
\usepackage{amssymb}
\usepackage{siunitx}
\usepackage{physics}
\usepackage{booktabs}
\usepackage{graphicx}
\usepackage{mathtools}
\usepackage{enumitem}
\pagestyle{headings}

\usepackage{geometry}
\geometry{
 a4paper,
 total={175mm,243mm},
 left=17mm,
 top=20mm,
}

\author{Antonio Tagliente, Davide Perrone, Matteo Vilucchio}

\title{Possibili soluzioni agli esami di "Complementi di Analisi"\\Corso di Laurea in Fisica\\ Università di Pisa}

\begin{document}
\maketitle
\section*{Compito 1}
\begin{itemize}
\item [1.]
Sia $f (x, y) = \abs{3x^{2}-2y^{4}}$ e sia D definito da:
\begin{equation*}
D:=\{(x, y)\in \mathbb{R}^{2} : x^{2} + y^{2} - 2x \leq 0\}.
\end{equation*}
Determinare estremo inferiore e superiore di f in D specificando se si tratta di
massimo e/o minimo e gli eventuali corrispondenti punti di massimo/minimo.\\
\emph{Svolgimento}\\
\emph{ERRORE NELLO SVOLGIMENTO, RICONTROLLARE I PUNTI DI TAGLIO}
f(x,y)=$ | 3x^2-2y^4|$
studiamo prima f nel caso in cui $3x^2 \ge 2y^4$.

$\nabla f=(6x,-8y)$ che si annula solo in (0,0).
D=$\{(x,y):x^2+y^2-2x \le 0)\}$

Sia $ g= x^2+y^2-2x=0$
$ \nabla g=(2(x-2),2y)$ che è uguale a zero in (1,0).
Procediamo ora con i moltiplicatori di Lagrange:
\begin{equation}
\begin{cases} 
 2\lambda (x-1)=9x\\2\lambda y=-4y^3 
\end{cases}
 \end{equation}
 Dal sistema troviamo $ \lambda=-4y^2$ che sostituita nella prima da: $ (y_1)^{2}~=~-\frac{x}{4x-1}$.
 
Considerando $g(x,y_1)$ troviamo $ x_{1,2}=\frac{5}{2},\frac{1}{2} $, la prima soluzione non è accetabile in quanto da un valore di $y^2$ negativo, la seconda da i valori di $ y=\pm \frac{\sqrt{3}}{2} $, tuttavia questi punti non sono accettabili in quanto danno un valore di f negativo (quando questa era supposta positiva.)


Consideriamo ora $f(x,y)=2y^4-3x^2$, usando di nuovo i moltiplicatori di lagrange otteniamo di nuovo i punti $ (\frac{1}{2}  , \pm \frac{\sqrt{3}}{2} )$ che questa volta vanno bene.
Deduciamo dunque da i punti stazionari trovati che f ha minimo in zero e massimo in 3 in due punti  su D.




 \item[2.] Sia $f : \mathbb{R}^{2} \rightarrow \mathbb{R}$ definita da:
\begin{equation*}
f(x,y)=\frac{x^{2}y^{3} + \sin(x^{2}y)}{1 + x^{4} + \abs{y}^{7}}
\end{equation*}
\begin{enumerate}[label=(\alph*)]
\item Provare che l'origine è un punto stazionario e classificarlo
\item Stabilire se $f$ ammette massimo e/o minimo su $\mathbb{R}^{2}$
\item Provare che $f$ ammette almeno 5 punti stazionari
\item (Bonus) Sia $Q_{R}=[R,+\infty[\times[R,+\infty[$ e sia $M(R)=sup_{Q_{R}} f(x,y)$. Calcolare al variare di $\alpha \in \mathbb{R}$:
\begin{equation*}
\lim_{R\to +\infty} R^{\alpha} M(R)
\end{equation*} 
\end{enumerate}
\emph{Svolgimento}
\begin{enumerate}[label=(\alph*)]
\item Sviluppando con Taylor attorno a 0 fino all'ordine 5 al numeratore, si ha:
\begin{equation}
f(x,y)\approx \frac{x^{2}y^{3} + x^{2}y}{1 + x^{4} + \abs{y}^{7}}.
\end{equation}
Sviluppando il denominatore si ottiene, all'ordine più basso,
\begin{equation}
f(x,y)\approx (x^{2}y)(1 - x^{4} + o((x^{2}+y^{2})^{2}))\approx x^{2}y
\end{equation}
che è nulla in $(0,0)$. L'origine non è né massimo né minimo, in quanto la curva $\gamma(t):\{x=t, y=t, -1<t<1\}$, passa per $(0,0)$ ed assume valori sia positivi che negativi in un intorno del punto, essendo dispari la variabile $y$ al numeratore.
\item Per stabilire se la funzione ammetta massimo o minimo, se ne studia il comportamento all'infinito.
\begin{equation}
\lim_{x^2+y^2\to+\infty} f(x,y).
\end{equation}
Prendendo la curva $\gamma_{1}(t):\{x=t, y=0\}$ e facendo tendere $t$ ad infinito, la funzione risulta costantemente pari a 0. Quindi il limite, se esiste, vale 0.
\begin{equation}
0 \leq \abs{f(x,y)} \leq \frac{x^{2}y^{3}}{x^{4}+ \abs{y}^{7}}.
\end{equation}
Cambiando variabile, ponendo $\abs{y}^{7}=t^{4}$, e $x^{2} + t^{2} = r^{2}$, (ricordando che $\sin(x)^{4} + \cos(x)^{4} \geq m >0$) si ottiene una stima dall'alto, ossia
\begin{equation}
0 \leq \frac{x^{2}y^{3}}{x^{4}+ \abs{y}^{7}} \leq \frac{r^{2+12/7}}{mr^{4}} \rightarrow 0.
\end{equation}
La funzione assume inoltre almeno un valore positivo (nel primo quadrante) ed uno negativo (nel quarto). Quindi, per il teorema di Weierstrass generalizzato, la funzione ammette sia massimo sia minimo.
\item Si giunge alla soluzione notando che $f(-x,y)=f(x,y)$, quindi i due punti trovati nel primo e nel quarto quadrante sono simmetrici a due punti nel terzo e nel secondo. Questi, assieme all'origine, sono 5 dei punti stazionari della funzione.
\item (bonus) Per risolvere il limite richiesto si studia la funzione sull'insieme dato, cercandone il $sup$, con il metodo dei moltiplicatori di Lagrange. Per $R \rightarrow \infty$ si avrà, definitivamente, che $\sin(x^2y)$ non influisce nello studio della funzione, ed i punti stazionari si troveranno al di fuori del rettangolo. Dunque si studia il comportamento della funzione sul bordo, ossia per $x=R, y \in [R,+\infty[$ e $y=R, x \in [R,+\infty[$.
Nel primo caso si ottiene:
\begin{equation}
\frac{R^{2}y^{3}}{1+R^{4}+y^{7}}
\end{equation}
che ha massimo per $y=0$ o $4y^{7}=3+3R^{4}$. Entrambi i risultati non sono accettabili, poichè fuori dal dominio. Per x si ottiene invece $x\approx R^{7/4}, y=R$. Quindi il limite diventa:
\begin{equation}
\lim_{R\to +\infty} \frac{R^{\alpha + 19/4}}{1+2R^{7}}
\end{equation}
che quindi tende a 0 per $\alpha < -9/4$, ad $1/2$ per $\alpha = -9/4$.  
\end{enumerate}






 \item[3.] Sia $V := \{(x, y, z)\in \mathbb{R}^{3} : x^{2} + y^{2} + z^{2} \leq 4, x^{2} + z^{2} \geq 1\}$. Calcolare
\begin{equation*}
\int_{V} \abs{y}\; dx\; dy \; dz.
\end{equation*}
\emph{Svolgimento}\\
Per svolgere l'integrale con il valore assoluto è possibile decomporre l'integrale in due parti, suddividendo il dominio. Si ha quindi
\begin{equation}
-\int_{V} y\; dx\; dy \; dz\: + \: 2\int_{V^{+}} y\; dx\; dy \; dz\:,
\end{equation}
con $V^{+} = \{(x, y, z)\in \mathbb{R}^{3} : x^{2} + y^{2} + z^{2} \leq 4,\; x^{2} + z^{2} \geq 1,\; y \geq 0 \}$ .\\
Il primo integrale è uguale a 0, poichè $f(x,-y)=-f(x,y)$ ed il dominio è simmetrico in y.
Per il secondo, invece, si passa alle coordinate cilindriche, $x^{2} + z^{2} = \rho^{2},\: y=Y$, e si integra utilizzando il teorema di Guldino, in quanto si ha a che fare con un solido di rotazione, e scrivendo l'insieme come normale rispetto all'asse $Y$, sul piano $(Y,\rho)$.\\
Dunque, trovato $A=(1,\sqrt{3})$ punto di intersezione, $D = \{(\rho, Y)\in \mathbb{R}^{2} : 1 \leq \rho^{2} \leq \sqrt{4 - Y^{2}},\; 0 \leq Y \geq \sqrt{3} \}$, e l'integrale diventa
\begin{equation}
4\pi\int_{0}^{\sqrt{3}} Y\; dY \int_{1}^{\sqrt{4-Y^{2}}} \rho d\rho\:,
\end{equation}
ossia:
\begin{equation}
2\pi\int_{0}^{\sqrt{3}} Y(3-Y^{2}) \; dY \: = \: \frac{\pi}{2}(6Y^{2} - Y^{4})\bigg{|}_{0}^{\sqrt{3}}\: = \:\frac{9\pi}{2}
\end{equation}





\item [4.]
Sia $F(x,y,z)=(x+y,x^2,z)$ e sia:
\begin{equation*}
S= \{(x,y,z):x^2+y^2+z^2+y^2z^2 \le 7 ,y \ge 0, z \ge 0 \}
\end{equation*}
orientata prendendo in $(2, 1, 1)$ la normale che punta verso le y negative. Calcolare
il flusso del rotore di F attraverso S.\\
\emph{Svolgimento}\\
Dobbiamo calcolare $ \int_{S} rot(F)\,dS $, la superficie S è orientata prendendo in (2,1,1) la normale che punta verso le y negative.
Per il teorema di stokes l'integrale è lo stesso su superfici che hanno lo stesso bordo; prendiamo le due superfici che si ottengono ponendo $y=0$ e $z =0$; $ rot(F)= (2x-1) \hat{z} $.
L'integrale su $S_1={x^2+y^2 = 7 y \ge 0}$ è nulllo in quanyto $S_1 $ ha normale (0,1,0) che ha prodotto scalare con $rot(F)$ nullo.
L'integrale su $ S_2=\{x^2+z^2=7, z \ge 0 \}$ è uguale a :
\begin{equation}
\int_{S_2}(2x-1)\,ds=\int_{0}^{\pi} \int_{0}^{\sqrt{7}} (2\rho \cos(\theta)-1) \rho d\rho d\theta=-\frac{7}{2}\pi
\end{equation}
\end{itemize}





\section*{Compito 2}
\begin{itemize}
\item[1.] Siano 
\begin{equation*}
S := \{ (x, y, z) \in \mathbb{R}^3 : x^2 + y^2 + x^2z^2 = 1, x\ge 0\} \quad f(x, y, z) =  x + y - z^2
\end{equation*}
Determinare $\inf_S f$ e $\sup_S f$ precisando se si tratta di minimo/massimo e gli eventuali corrispondenti punti di minimo/massimo.\\
\emph{Soluzione}\\
L'insieme $S$ non è limitato in quanto, nel caso $x = 0$ la $y$ è fissata a $\pm 1$ ma la $z$ può assumere tutti i valori in $\mathbb{R}$.
Riscrivendo la condizione dell'insieme otteniamo che $y^2 = 1-x^2 (1-z^2)$, siccome $x^2 (1-z^2)$ è una quantità sempre positiva $y$ sarà sempre limitato tra $-1 \le y \le 1$. Per $x$ abbiamo che $x^2 = \frac{1-y^2}{1+ z^2}$ e quindi risulta che $0 \le x \le 1$. Le $z$ risultano illimitate.
Il limite all'infinito nell'insieme $S$ risulta:
\begin{equation}
\lim\limits_{\norm{(x,y)} \rightarrow +\infty} x + y - z^2 \le \lim\limits_{\norm{(x,y)} \rightarrow +\infty} 2 - z^2 = - \infty 
\end{equation}
Il limite su $S$ è $- \infty$, quindi $\inf_{S} f = -\infty$. Per il teorema di Weierstrass generalizzato sappiamo anche che esiste un punto di massimo in $S$.\\
Per trovare l'inf avremmo potuto considerare la curva $(0, 1, t)$ con $t \in \mathbb{R}$, interamente contenuta in $S$, e trovare che il limite è $-\infty$, ma questo non sarebbe bastato ad applicare Weierstrass.\\
La funzione avrà sicuramente il massimo per $z=0$, quindi basta studiare $h(x,y)=f(x,y,0)=x+y$ su $S' = \{ (x,y) \in \mathbb{R}^2 : x^2 + y^2 =1, x\ge 0 \}$. Senza scomodare i moltiplicatori di Lagrange sostituiamo il bordo nella funzione e studiamo la funzione di una varaibile che troviamo.
\begin{equation}
g(y) = h(\sqrt{1-y^2}, y) = y + \sqrt{1-y^2} \quad \dv{g}{y} = y - \frac{y}{\sqrt{1-y^2}} = 0
\end{equation}
La derivata si annulla in $y = \pm \frac{\sqrt{2}}{2}$ e i corrispondenti punti per $h(x, y)$ sono $(\frac{\sqrt{2}}{2}, \frac{\sqrt{2}}{2})$ e $(\frac{\sqrt{2}}{2}, -\frac{\sqrt{2}}{2})$. Il massimo si ha per il primo dei due punti e la funzione vale $\sqrt{2}$. 
I punti di taglio per $h$ sono $(0, \pm 1)$ e la funzione vale $\pm 1$.
Quindi il $\max_{S} f = \sqrt{2}$.


È VERAMENTE NECESSARIO FARE IL GRDIENTE?
Controlliamo se si annulla il gradiente di $f$ in $S$.
\begin{equation}
\grad f = \begin{pmatrix}
1 \\ 1 \\ -2z
\end{pmatrix} \neq \begin{pmatrix}
0 \\ 0\\ 0
\end{pmatrix} \quad \quad \forall (x,y,z) \in \mathbb{R}^3
\end{equation}



\item[2.] Sia $Q := [1,+\infty[\times [1,+\infty[$.
\begin{enumerate}[label=(\alph*)]
\item Stabilire se convergono:
\begin{equation*}
\int_{Q}\frac{\arctan(xy)}{x^{2}+y^{2}}dxdy,\quad \int_{Q}\frac{\arctan(xy)}{x^{2}+y^{4}}dxdy
\end{equation*}
\item (Bonus) Sia $Q_{n} = [n,2n]\times [n,2n]$. Calcolare, al variare di $\alpha \in \mathbb{R}$:
\begin{equation*}
\lim_{n\to +\infty}n^{\alpha}\int_{Q_{n}}\frac{\arctan(xy)}{x^{2}+y^{4}}dxdy
\end{equation*}
\end{enumerate}
\emph{soluzione}\\
\begin{enumerate}[label=(\alph*)]
\item Il numeratore è crescente per entrambe le variabili, quindi, nell'insieme, si ha $\arctan(xy) \geq \arctan(1) = \pi/4$.
Quindi vale:
\begin{equation}
\frac{\pi}{4}\int_{Q}\frac{1}{x^{2}+y^{2}}dxdy\leq\int_{Q}\frac{\arctan(xy)}{x^{2}+y^{2}}dxdy
\end{equation}
Passando alle coordinate polari e restringendo Q in modo da avere $\pi/6\leq\theta\leq\pi/3$ (per non avere problemi con gli angoli), l'integrale diverge.\\
Per il secondo si ha, maggiorando l'$\arctan(xy)$:
\begin{equation}
\int_{Q}\frac{\arctan(xy)}{x^{2}+y^{4}}dxdy \leq \int_{Q}\frac{\pi}{4}\frac{1}{x^{2}+y^{4}}dxdy.
\end{equation}
A questo punto, cambiando variabile, ponendo $x^{2}=z^{4}$ ed infine passando in polari con $y^{2}+z^{2}=r^{2}$:
\begin{equation}
\int_{Q}\frac{\pi}{4}\frac{1}{x^{2}+y^{4}}dxdy\leq\int_{0}^{\pi/4}d\theta\int_{2}^{+\infty}\frac{2r^{2}}{mr^{4}} < +\infty
\end{equation}
dove uno dei fattori $r$ deriva dal primo cambio di variabile, il secondo dal passaggio in polari, con $m>0$ minorazione di $\sin(\theta)^{4}+\cos(\theta)^{4}$
\item cambio di variabile?
\end{enumerate}


\item[3.] Sia T il triangolo del piano $xy$ di vertici $(1, 0)$, $(2, 0)$ e $(3, 2)$ sia V il solido
ottenuto da una rotazione completa di T intorno all’asse y. Calcolare il volume e
le coordinate del baricentro di V.\\
\emph{soluzione}\\
La figura è un solido di rotazione, quindi, per calcolarne volume e baricentro, si utilizza il teorema di Guldino. Si sceglie quindi l'asse $y$ come privilegiato e si integra come insieme normale rispetto a quest'asse. La figura è compresa tra le rette $y=x-1$, $y=2x-4$, $y=0$. Quindi:
\begin{equation}
V \; = \; 2\pi\int_{0}^{2}dy\int_{y+1}^{(y+4)/2}xdx \; = \; 3\pi\int_{0}^{2}(1-y^2)dy\; = \; 4\pi
\end{equation}
Per le coordinate del baricentro si studia solo la coordinata y, in quanto le altre sono uguali a 0 per simmetria. Quindi:
\begin{equation}
\frac{2\pi}{V}\int_{0}^{2}dy\int_{y+1}^{(y+4)/2}yxdx \; = \; \frac{3}{4}\int_{0}^{2}y(1-y^2)dy\; = \; \frac{3}{4}
\end{equation}
\item[4.] Sia $\gamma$ la curva del piano $(x, y)$ definita da: $\gamma(t) = (t^{2} - 2t^{3},t-t^{2} )$, con $0 \leq t \leq 3/2$.
\begin{enumerate}[label=(\alph*)]
\item Determinare se $\gamma$ è semplice e farne un disegno approssimativo.
\item Determinare le intersezioni tra $\gamma$ e la retta $6y = x$.
\item Sia D la regione di piano racchiusa da $\gamma \cup \{6y = x\}$. Calcolare l’area di D.
\end{enumerate}
\emph{Soluzione}\\
\begin{enumerate}[label=(\alph*)]
\item Per dimostrare che $\gamma$ è semplice basta trovare una funzione $G(x(t),y(t))$ monotona, ad esempio $G(x,y)=x-y^2$, che, calcolata nelle componenti della curva, diventa $G(x(t),y(t))=-t^4$, che è monotona nell'intervallo di $t$ considerato
\item Le intersezioni si trovano sostituendo nella retta $6y=x$ le componenti della curva. Si ritrovano quindi intersezioni per $(x,y)=(0,0)$ e $(x,y)=(-9/2,-3/4)$.
\item Per calcolare l'area si può utilizzare il teorema di Gauss Green, poichè
\begin{equation}
area(D)\; = \; \int_{D}1dxdy \; = \; \int_{\partial^{+}D}F_{x}dy; \: divF=1
\end{equation}
con $F_{x}$ componente x della funzione vettoriale $F$ e $\partial^{+}D$ bordo del dominio costituito dalle due curve $\gamma$ e $\gamma_{1}$:
\begin{equation}
\gamma=
\begin{cases}
x=t^2 - 2t^3 \\
y= t - t^2 \\
0 \leq t \leq 3/2
\end{cases}
\gamma_{1}=
\begin{cases}
x=6t\\
y=t\\
-\frac{3}{4} \leq t \leq 0.
\end{cases}
\end{equation}
L'integrale quindi diventa, scegliendo $F=(x,0,0)$:
\begin{equation}
\int_{0}^{3/2}(t^2 - 2t^3)(1-2t)dt\; + \; \int_{-3/4}^{0}6tdt \;=
\end{equation}
\begin{equation*}
= \;t^3\bigg(\frac{1}{3} - t + \frac{4}{5}t^2 \bigg) \bigg|_{0}^{3/2} + 3t^2\bigg|_{-3/4}^{0} = \frac{9}{20}
\end{equation*}
\end{enumerate}
\end{itemize}


\section*{Compito 3}
\begin{itemize}
\item [1.]
\item [2.]
\item [3.] Sia $B$ la sfera di $\mathbb{R}^3$ di centro $(1,0,2)$ e raggio 2. Calcolare
\begin{equation*}
\int_{B} |x| dxdydz
\end{equation*}
\emph{soluzione}\\
Per prima cosa facciamo una traslazione al centro della sfera ed introduciamo le nuove variabili come $t = x -1$, $u = y$ e $v = z -2$. Cos' l'integrale diventa:
\begin{equation}
\int_{B'} |t + 1| dtdudv
\end{equation}
e il dominio diventa la sfera di raggio 2 centrata nell'origine. DIvidiamo il dominio nella parte in cui l'argomento del valore assoluto è positivo e quella in cui è negativo, rispettivamente saranno \\$A_{+} = \{(t, u,v) \in \mathbb{R}^3 : t^2 + u^2 + v^2 \le 2, \: t \ge -1\}$ e \\$A_{-} = \{(t,u,v) \in \mathbb{R}^3 :t^2 + u^2 + v^2 \le 2, \: t \le -1\}$.\\
Possiamo quindi scegliere diverse strade: sommare i due integrali sui due dominii con i segni cambiati, fare l'integrale su $A$ e sommare/sottrarre due volte l'integrale su uno dei due sottodominii. Noi calcoleremo:
\begin{equation}
-\int_A t + 1 \: dtdudv + 2 \int_{A_{+}} t + 1 \: dtdudv
\end{equation}
Nel primo integrale il contributo dato da $t$ è nullo in quanto è una funzione "dispari" su un dominio simmetrico. Rimane quindi solamente il volume di una sfera di raggio 2.
\begin{equation}
\int_A t+1 \: dtdudv = \frac{32}{3} \pi
\end{equation}
Per integrare il secondo pezzo utilizziamo un sistema di coodinate cilindriche con asse privilegiato $t$. 
\begin{equation}
\int_{A_{+}} t+ 1 \: dtdudv = \int_{-1}^{2}dt \int_0^{2\pi} d\theta \int_{0}^{\sqrt{4-t^2}} (t+1) \rho d\rho =
\end{equation}
\begin{equation*}
=\pi \int_{-1}^{2} (4- t^2)(t + 1) dt = \pi \int_{-1}^{2} t4 - t^3 + 4 -t^2 dt = \pi \Big(2t^2 -\frac{t^4}{4} + 4t -\frac{t^3}{3}\Big)\Big|_{-1}^{2} = \pi \frac{45}{4}
\end{equation*}
Per il calcolo dell' integrale risulta:
\begin{equation}
-\frac{32}{3} \pi + 2 \Big( \frac{45}{4} \pi \Big) = \frac{71}{6}\pi
\end{equation}
\item [4.] Si consideri per $\alpha $ la forma differenziale

\begin{equation*}
\omega_{\alpha} = e^{\alpha xy} (xy + y^2 + 1) dx + e^{\alpha xy} (x^2 + yx + 1) dy + dz
\end{equation*} 
e sia $\gamma (t) = (\cos^3t, \sin t, \cos t)$ con $0 < t < \pi$.
Calcolare per $\alpha = 0$ e $\alpha = 1$
\begin{equation*}
\int_{\gamma} \omega_{\alpha}
\end{equation*}
\emph{soluzione}\\
Ponendo le varie componenti della forma come $f_1, f_2, f_3$ verifichiamo l chiusura della forma con le condizioni:
\begin{equation}
\begin{cases}
f_{1y} = f_{2x} \\
f_{2z} = f_{3y} \\
f_{3x} = f_{1z}
\end{cases}
\end{equation}
La seconda e la terza condizione sono sempre verificate per $\forall \alpha \in \mathbb{R}$. Pe la prima condizione abbiamo che:
\begin{equation}
\alpha x e^{\alpha xy} (xy + y^2 + 1) + e^{\alpha xy} (x + 2y) = \alpha y e^{\alpha xy} (x^2 + yx + 1) + e^{\alpha xy} (2x + y)
\end{equation}
\begin{equation*}
(\alpha +1)x + 2y = 2x + (\alpha +1)y
\end{equation*}
la forma risulta chiusa solamente per $\alpha = 1$.\\

Siccome per $\alpha = 1$ $\omega$ è una forma chiusa possiamo sostituire a $\gamma (t)$ una curva continua e semplice con gli stessi etremi. Prendiamo quindi $\gamma_1 (t) = (-t,0,-t)$ con $-1 \le t \le 1$. Calcoliamo adesso l'integrale di linea:
\begin{equation}
\int_{\gamma_1} \omega_1 = \int_{\gamma_1} \limits\sum_{i=1}^3 f_i(x(t), y(t), z(t)) x_i'(t) dt = \int_{-1}^{1} t - 2 dt = -4
\end{equation}
\\
Per il primo integrale potremmo riparametrizzare la curva, ad esempio ponendo $u = \cos t$, ma non si guadagnerebbe molto. Svolgiamo l'integrale comunque:
\begin{align}
\int_{\gamma} \omega_0 &= \int_{0}^{\pi} - 3(\cos^3 t \sin t + \sin^2 t + 1)\cos^3 t \sin t + \nonumber\\
&+ (\cos^6 t + \cos^3 t \sin t + 1)\cos t - \sin t \quad dt = \nonumber\\
&= \int_{0}^{\pi} -3(\cos^5 t \sin^2 t + \cos^2 t \sin t - \cos^4 t \sin t + \cos^2 t \sin t) + \nonumber\\
&+ \cos^7 t + \cos^4 t \sin t + \cos t -\sin t \quad dt 
\end{align}
Siccome stiamo integrando da $0$ a $\pi$ possiamo già dire che tutti gli integrali con potenze dispari di $\cos t$ saranno 0. Per gli altri abbiamo la derivata del coseno e si ntegrano facilmente dando:
\begin{align}
\int_{\gamma} \omega_0 &= \int_{0}^{\pi} -6\cos^2 t \sin t +4\cos^4 t \sin t - \sin t \: dt = \nonumber\\
&= 2\cos^3 t - \frac{4}{5} \cos^5 t + \cos t \Big|_{0}^{\pi} = -\frac{22}{5}
\end{align}

\end{itemize}



\end{document}
